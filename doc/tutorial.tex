\documentclass[11pt]{article}

\title{\textsf{molsim} tutorial}
\author{J.S. Hansen}
\date{Februaray 2022, v. 0.9}
  
\begin{document}

\maketitle

\section{Introduction}

\textsf{molsim} is a GNU Octave/Matlab package for molecular dynamics simulation
library. \textsf{molsim} supports simulations of
\begin{itemize}
\item Standard Lennard-Jones systems (solid, liquids, gasses, etc)
\item Molecular systems with bond, angle, and torsion potentials 
\item Confined flow systems, eg., Couette and Poiseuille flows
\item Charged systems using shifted force and Wolf methods
\item Dissipative particle dynamics systems
\item and more
\end{itemize}
The package also supports a series of run-time sampling functionalities.

\bigskip
\noindent \textsf{molsim} is basically a wrapper for the \textsf{seplib}
library, which is a light-weight flexible molecular dynamics simulation library
written in ISO-C99. The library is CPU-based and offers shared memory
parallisation; this parallisaton is supported by the \textsf{molsim}
package. The algorithms used in \textsf{seplib} is based on the books by Allen
\& Tildesley, Rapaport, Frenkel \& Smith, and R. Sadus, see
Ref. \cite{seplib:books}.

\bigskip
\noindent In this text
\begin{verbatim}
>> 
\end{verbatim}
indicates GNU Octave or Matlab command prompt. This 
\begin{verbatim}
$ 
\end{verbatim}
indicates the shell prompt.

\section{Installation}
\subsection{GNU Octave}
GNU Octave's package manager offers a very easy installation. From

\begin{verbatim}
https://github.com/jesperschmidthansen/molsim/
\end{verbatim}

\noindent download and save the current release
\verb!molsim-<version>.tar.gz! in a directory of your choice. Start GNU
Octave and if needed change directory to the directory where the file is saved.
\begin{verbatim}
>> pkg install molsim-<version>.tar.gz 
\end{verbatim}
Check contact by
\begin{verbatim}
>> molsim('hello')
Hello 
\end{verbatim}
In case this fails, check the path where \textsf{molsim} is install by
\begin{verbatim}
>> pkg list molsim
\end{verbatim}
If the path is not in your GNU Octave search path add this using the
\verb!addpath! command.

\subsection{Matlab}
From
\begin{verbatim}
https://github.com/jesperschmidthansen/seplib/
\end{verbatim}
\noindent download and save the current release \verb!seplib-<version>.tar.gz!
in a directory of your choice. Unpack, configure and build the library
\begin{verbatim}
$ tar zxvf seplib-<version>.tar.gz
$ cd seplib
$ ./configure
$ make
$ cd octave
\end{verbatim}
To build the \textsf{mex}-file enter Matlab
\begin{verbatim}
$ matlab -nodesktop
\end{verbatim}
Then build the 
\begin{verbatim}
>> buildmex
\end{verbatim}
Depending on the system this will build a \textsf{molsim.mex<archtype>}
file. You can copy this file to a directory in your Matlab searce path.

\section{First quick example: The Lennard-Jones liquid}
Listing 1 shows the simplest script simulating a standard Lennard-Jones system.

\noindent \textbf{Listing 1}
\begin{verbatim}
1: cutoff = 2.5; epsilon = 1.0; sigma = 1.0; aw=1.0;

2: molsim('set', 'lattice', [10 10 10], [12 12 12]);
3: molsim('load', 'xyz', 'start.xyz');

4: for n=1:10000

5:   molsim('reset');
6:   molsim('calcforce', 'lj', 'AA', cutoff, sigma, epsilon, aw);
7:   molsim('integrate', 'leapfrog');
 
8: end

9: molsim('clear');
\end{verbatim}

\noindent Line 1: Specification of the Lennard-Jones interaction potential.

\noindent Line 2: Writes initial particle positions and velocities to default file
\textsf{start.xyz}. System dimension is set to $10 \times 10 \times 10$, and box
lengths to $12 \times 12 \times 12$. 

\noindent Line 3: The initial configuration is loaded

\noindent Line 4 \& 8: Molecular dynamics main loop. 10$^4$ iterations are performed

\noindent Line 5: Everything reset

\noindent Line 6: Calculate force between particles of type A using the interaction
parameter specifications

\noindent Line 7: Integrate forward in time

\noindent Line 9: Free all memory allocated.

\subsection{Extending the example}

\section{The force field and more examples}
\section{Confined systems}
\section{Sampling}
\section{The two parallisation paradigms}


\end{document}
